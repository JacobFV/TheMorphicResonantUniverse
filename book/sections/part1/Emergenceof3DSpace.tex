
\section{Dimensional reduction and the emergence of a low-dimensional subspace}
\subsection{Spectral dimension and the scale-dependent dimensionality}
The emergence of 3D space from the USMS is a process of dimensional reduction, which leads to the formation of a low-dimensional subspace that has a well-defined dimensionality and geometry. The dimensionality of the subspace is measured by the spectral dimension, which is a scale-dependent quantity that reflects the connectivity and topology of the states. The spectral dimension is defined as the scaling exponent of the return probability of a random walk on the states, which measures the probability of returning to the initial state after a given number of steps.

\begin{tcolorbox}[colback=blue!5!white,colframe=blue!75!black,title=New terms]
\begin{description}
\item[Spectral dimension:] A measure of the effective dimensionality of a space or a network, based on the scaling behavior of the diffusion or the random walk on the space. The spectral dimension $d_s$ is defined as the exponent of the return probability of a random walk, $P(t) \sim t^{-d_s/2}$, where $t$ is the number of steps of the walk. The spectral dimension can be different from the topological dimension of the space, and it can vary with the scale and the resolution of the diffusion process.
\item[Return probability:] The probability of a random walker to return to its initial position after a given number of steps. The return probability is a measure of the recurrence and the connectivity of the space, and it depends on the dimensionality and the topology of the space. In a $d$-dimensional Euclidean space, the return probability scales as $P(t) \sim t^{-d/2}$, which reflects the diffusive nature of the random walk.
\end{description}
\end{tcolorbox}

\subsection{Renormalization group flow and the infrared limit}
The dimensional reduction is driven by the renormalization group flow, which is a process of coarse-graining and rescaling of the states, that leads to the elimination of the high-frequency and short-range degrees of freedom. The renormalization group flow is characterized by a set of fixed points, which are the scale-invariant and self-similar states that are stable under the coarse-graining and rescaling transformations. The infrared limit of the renormalization group flow corresponds to the low-energy and long-range limit of the states, which is characterized by a reduced dimensionality and a simplified geometry.

\begin{tcolorbox}[colback=green!5!white,colframe=green!75!black,title=Question]
How does the dimensional reduction process relate to the holographic principle and the AdS/CFT correspondence?
\tcblower
The dimensional reduction process is closely related to the holographic principle and the AdS/CFT correspondence, which are two of the most important developments in theoretical physics in recent decades.

The holographic principle states that the information and the degrees of freedom of a region of space can be encoded on the boundary of the region, with a density that is proportional to the area of the boundary. This implies that the effective dimensionality of the space is reduced by one, and that the boundary theory is a hologram of the bulk theory. The holographic principle was inspired by the properties of black holes, which have an entropy that is proportional to their surface area, and it has been generalized to other gravitational systems, such as the anti-de Sitter (AdS) space.

The AdS/CFT correspondence is a concrete realization of the holographic principle, which states that a gravitational theory in a $d$-dimensional AdS space is equivalent to a conformal field theory (CFT) in a $(d-1)$-dimensional flat space, which lives on the boundary of the AdS space. The AdS/CFT correspondence provides a duality between the bulk and the boundary theories, which allows to calculate the observables of one theory in terms of the other, and to study the emergence of space and gravity from the perspective of the boundary theory.

In the context of our framework, the dimensional reduction process can be seen as a manifestation of the holographic principle, which relates the degrees of freedom of the USMS to the degrees of freedom of the emergent 3D space. The renormalization group flow can be interpreted as a holographic mapping between the bulk and the boundary theories, which preserves the scale invariance and the self-similarity of the states. The infrared limit of the renormalization group flow corresponds to the low-energy and long-range limit of the boundary theory, which is described by an effective field theory that captures the emergent properties and symmetries of the bulk space.

Thus, the dimensional reduction process provides a bridge between the abstract concept of the USMS and the concrete realization of the holographic principle and the AdS/CFT correspondence, and it offers a new perspective on the emergence of space and gravity from the fundamental degrees of freedom of the universe.
\end{tcolorbox}

\section{Pseudo-rectilinear structure and the emergence of Euclidean geometry}
\subsection{Lattice-like connectivity and the emergence of translation symmetry}
The emergent 3D space has a pseudo-rectilinear structure, which is characterized by a lattice-like connectivity and a translation symmetry. The lattice-like connectivity is a consequence of the morphic resonance, which leads to the formation of a regular and periodic arrangement of the states , that minimizes the distance and maximizes the similarity between them. The translation symmetry is a consequence of the homogeneity and isotropy of the emergent space, which means that the properties and patterns of the states are invariant under translations and rotations.

\begin{tcolorbox}[colback=blue!5!white,colframe=blue!75!black,title=New terms]
\begin{description}
\item[Lattice:] A regular and periodic arrangement of points or objects in space, which forms a symmetric and repeating pattern. Lattices are characterized by their unit cell, which is the smallest repeating unit of the pattern, and by their symmetry group, which describes the set of transformations that leave the lattice invariant.
\item[Translation symmetry:] A symmetry of a system or a pattern under a translation of the coordinates by a constant vector. Translation symmetry implies that the properties and the equations of the system are invariant under a shift of the origin, and that the system has a conserved quantity, which is the momentum.
\end{description}
\end{tcolorbox}

\subsection{Approximate rotational symmetry and the emergence of isotropy}
The emergent 3D space has an approximate rotational symmetry, which means that the properties and patterns of the states are invariant under rotations, up to a certain accuracy and resolution. The rotational symmetry is a consequence of the morphic resonance, which leads to the formation of spherically symmetric and isotropic structures, that minimize the anisotropy and maximize the similarity between the states. The approximate nature of the rotational symmetry is a consequence of the finite size and resolution of the emergent space, which leads to a small but non-zero anisotropy and inhomogeneity.

\begin{tcolorbox}[colback=green!5!white,colframe=green!75!black,title=Question]
How does the pseudo-rectilinear structure of the emergent 3D space relate to the observed geometry of the universe?
\tcblower
The pseudo-rectilinear structure of the emergent 3D space is consistent with the observed geometry of the universe on large scales, which is described by the Friedmann-Lemaître-Robertson-Walker (FLRW) metric. The FLRW metric is a solution of the Einstein field equations of general relativity, which assumes that the universe is homogeneous and isotropic on large scales, and that it has a constant curvature that depends on the density and the composition of the matter and energy in the universe.

The homogeneity and isotropy of the FLRW metric are reflected in the translation and rotational symmetries of the emergent 3D space, which ensure that the properties and patterns of the states are invariant under translations and rotations, up to a certain accuracy and resolution. The constant curvature of the FLRW metric is related to the lattice-like connectivity of the emergent 3D space, which determines the intrinsic geometry and topology of the space, and which can be positive, negative, or zero, depending on the balance between the attractive and repulsive forces in the universe.

The pseudo-rectilinear structure of the emergent 3D space also provides a natural explanation for the observed flatness and uniformity of the universe, which are two of the main puzzles of modern cosmology. The flatness problem refers to the fact that the observed curvature of the universe is very close to zero, which requires an extreme fine-tuning of the initial conditions of the universe. The uniformity problem refers to the fact that the observed temperature and density fluctuations of the cosmic microwave background are very small and homogeneous, which requires a mechanism for the synchronization and equilibration of the different regions of the universe.

In the context of our framework, the flatness and uniformity of the universe can be seen as a consequence of the morphic resonance and the renormalization group flow, which lead to the formation of a regular and periodic lattice of states, that minimizes the curvature and the inhomogeneity of the emergent space. The morphic resonance provides a non-local and non-causal mechanism for the synchronization and equilibration of the different regions of the universe, which can explain the observed uniformity of the cosmic microwave background. The renormalization group flow provides a dynamical mechanism for the flattening and the smoothing of the emergent space, which can explain the observed flatness of the universe, without requiring any fine-tuning of the initial conditions.

Thus, the pseudo-rectilinear structure of the emergent 3D space provides a unified and consistent description of the observed geometry and topology of the universe, and it offers a new perspective on the origin and the evolution of the cosmic structures, from the fundamental degrees of freedom of the USMS.
\end{tcolorbox}

\section{Topological and geometric phases}
\subsection{Topological phase transitions and the emergence of non-trivial topological invariants}
The emergence of 3D space is accompanied by a series of topological phase transitions, which lead to the formation of non-trivial topological invariants, such as the Euler characteristic, the Betti numbers, and the Chern classes. The topological phase transitions are driven by the changes in the connectivity and topology of the states, which lead to the formation of holes, voids, and other non-trivial structures. The non-trivial topological invariants are a consequence of the global and non-local properties of the emergent space, which cannot be described by the local and classical geometry.

\begin{tcolorbox}[colback=blue!5!white,colframe=blue!75!black,title=New terms]
\begin{description}
\item[Topological invariant:] A quantity that characterizes the global and non-local properties of a space or a manifold, and that remains unchanged under continuous deformations or transformations of the space. Examples of topological invariants include the Euler characteristic, which measures the number of vertices, edges, and faces of a polyhedron, and the Betti numbers, which measure the number of connected components, holes, and voids of a manifold.
\item[Topological phase transition:] A phase transition that is driven by the changes in the topology of the system, and that is characterized by the appearance or disappearance of non-trivial topological invariants. Topological phase transitions are different from the usual thermodynamic phase transitions, which are driven by the changes in the local order parameters, and they can occur at zero temperature and without any symmetry breaking.
\end{description}
\end{tcolorbox}

\subsection{Geometric phase transitions and the emergence of a non-trivial metric tensor}
The emergence of 3D space is also accompanied by a series of geometric phase transitions, which lead to the formation of a non-trivial metric tensor, that describes the intrinsic curvature and geometry of the space. The geometric phase transitions are driven by the changes in the curvature and topology of the states, which lead to the formation of a curved and non-Euclidean geometry. The non-trivial metric tensor is a consequence of the non-linear and self-interacting nature of the emergent space, which cannot be described by the linear and flat geometry of Euclidean space.

\begin{figure}[h]
    \centering
    % TODO:
    % \includegraphics[width=0.8\textwidth]{topological_phases_diagram.png}
    \caption{Schematic representation of the topological and geometric phase transitions in the emergence of 3D space. The topological phase transitions are characterized by the appearance of non-trivial topological invariants, such as the Euler characteristic $\chi$ and the Betti numbers $b_i$, which measure the global and non-local properties of the emergent space. The geometric phase transitions are characterized by the appearance of a non-trivial metric tensor $g_{\mu\nu}$, which measures the intrinsic curvature and geometry of the emergent space. The figure shows the evolution of the topological and geometric properties of the emergent space as a function of the scale and resolution of the measurements, from the UV to the IR limit.}
    \label{fig:topological_phases}
\end{figure}
