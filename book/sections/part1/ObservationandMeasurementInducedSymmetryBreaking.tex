
% this into needs to sound more solemn and genuine
Have you ever wondered why the world around us looks so classical and deterministic, even though the fundamental laws of physics are quantum and probabilistic? Why do we always see particles in definite positions and states, even though they can exist in superpositions and entanglements according to quantum mechanics?

The answer lies in the process of observation and measurement, which is the act of extracting information from a quantum system by interacting with it using a classical apparatus. As we will see in this chapter, the measurement process is not a passive and neutral act, but a dynamic and transformative one, which can change the state of the system and break its symmetries in a dramatic and irreversible way.

In particular, we will explore how the measurement process can induce a spontaneous symmetry breaking of the USMS, leading to the emergence of classical structures and geometries, and the reduction of the fundamental symmetries of the space. We will also see how the measurement process is intimately related to the phenomenon of decoherence, which is the loss of coherence and entanglement between the states of the system due to its interaction with the environment.

But what exactly is a measurement in quantum mechanics, and how does it differ from a classical measurement? And what is the role of the observer in the measurement process, and how does it relate to the problem of the interpretation of quantum mechanics?

These are some of the deep and fascinating questions that we will explore in this chapter, using the tools of quantum measurement theory and the concept of decoherence. We will see how the measurement process can be described mathematically using the von Neumann projection postulate and the Born rule, and how it can lead to the collapse of the wave function and the emergence of classical probabilities.

We will also discuss some of the conceptual and philosophical implications of the measurement process, such as the nature of reality and the role of consciousness in the universe. And we will see how the measurement-induced symmetry breaking of the USMS can provide a new perspective on the emergence of classicality from the quantum realm, and on the origin of the arrow of time and the second law of thermodynamics.

So, get ready to dive into the strange and wonderful world of quantum measurement theory, and to explore the mysteries of the observer and the observed!

\section{Quantum measurement theory and wave function collapse}
\subsection{Von Neumann's projection postulate}
The observation and measurement of the USMS is described by the quantum measurement theory, which is based on the von Neumann's projection postulate. According to the projection postulate, a measurement of an observable $A$ on a state $\ket{\psi}$ results in the collapse of the state onto one of the eigenstates $\ket{a_i}$ of $A$, with a probability given by the Born rule $p_i = |\braket{a_i|\psi}|^2$. The collapsed state $\ket{a_i}$ is an eigenstate of $A$ with an eigenvalue $a_i$, which represents the outcome of the measurement.

\begin{tcolorbox}[colback=blue!5!white,colframe=blue!75!black,title=New terms]
\begin{description}
\item[Observable:] A physical quantity that can be measured, represented by a Hermitian operator $A$ acting on the Hilbert space of the system. The eigenstates of $A$ form a complete orthonormal basis, and the eigenvalues of $A$ represent the possible outcomes of the measurement.
\item[Projection postulate:] A fundamental postulate of quantum mechanics, which states that a measurement of an observable $A$ on a state $\ket{\psi}$ projects the state onto one of the eigenstates of $A$, with a probability given by the Born rule. The projection postulate is a non-unitary and irreversible process, which is distinct from the unitary evolution of the state under the Schrödinger equation.
\end{description}
\end{tcolorbox}

\subsection{Born's rule and probability interpretation}
The Born rule is a fundamental postulate of quantum mechanics, which relates the probability of a measurement outcome to the inner product between the state and the eigenstates of the observable. The probability interpretation of the Born rule implies that the measurement process is inherently random and non-deterministic, and that the outcome of a measurement cannot be predicted with certainty. The Born rule also implies that the measurement process is non-unitary and irreversible, as it leads to the collapse of the state and the loss of information about the original state.

\begin{tcolorbox}[colback=green!5!white,colframe=green!75!black,title=Question]
What is the physical meaning of the wave function collapse? Is it a real process or just a mathematical artifact?
\tcblower
The nature of the wave function collapse is a controversial and unresolved issue in the foundations of quantum mechanics. There are different interpretations of the collapse process, ranging from purely epistemic ones (e.g., the Copenhagen interpretation), which view the collapse as a subjective update of the observer's knowledge, to objective ones (e.g., the spontaneous collapse theories), which view the collapse as a real physical process that occurs independently of the observer. In the context of our framework, the wave function collapse is a real process that is induced by the interaction between the USMS and the measurement apparatus, and that leads to the breaking of the symmetries of the USMS and the emergence of classical structures and geometries. The collapse process is not instantaneous, but is a gradual process that is described by the decoherence of the state and the suppression of the off-diagonal elements of the density matrix.
\end{tcolorbox}

\section{Decoherence and the emergence of classical states}
\subsection{Environmental interaction and the loss of coherence}
The measurement process is not an isolated event, but it involves the interaction between the USMS and the environment, which leads to the decoherence of the state and the loss of coherence between its components. The decoherence process is described by the Lindblad equation, which is a master equation that governs the evolution of the density matrix $\rho$ under the influence of the environment. The Lindblad equation includes a dissipative term that describes the decay of the off-diagonal elements of $\rho$, which represent the quantum correlations and the coherence between the states.

\begin{tcolorbox}[colback=blue!5!white,colframe=blue!75!black,title=New terms]
\begin{description}
\item[Decoherence:] The process by which a quantum system loses its coherence and becomes classical due to its interaction with the environment. Decoherence is a consequence of the entanglement between the system and the environment, which leads to the suppression of the off-diagonal elements of the density matrix and the emergence of classical probabilities and correlations.
\item[Lindblad equation:] A master equation that describes the non-unitary evolution of the density matrix of an open quantum system, which is coupled to an environment. The Lindblad equation has the form $\frac{d\rho}{dt} = -i[H,\rho] + \sum_i \gamma_i (L_i \rho L_i^\dagger - \frac{1}{2}\{L_i^\dagger L_i,\rho\})$, where $H$ is the Hamiltonian of the system, $L_i$ are the Lindblad operators that represent the interaction with the environment, and $\gamma_i$ are the decay rates associated with each Lindblad operator.
\end{description}
\end{tcolorbox}

\subsection{Pointer states and the preferred basis problem}
The decoherence process leads to the emergence of a preferred basis of states, which are the pointer states that are stable under the environmental interaction. The pointer states are the eigenstates of the observable that is being measured, and they correspond to the classical states that have a well-defined value of the observable. The preferred basis problem is the question of how the pointer states are selected from the infinitely many possible bases of the Hilbert space, and how they depend on the specific form of the environmental interaction.

\begin{tcolorbox}[colback=green!5!white,colframe=green!75!black,title=Question]
How does the decoherence process explain the emergence of classical reality from the quantum world?
\tcblower
The decoherence process provides a mechanism for the emergence of classical reality from the quantum world by suppressing the quantum coherence and entanglement between the states of the system. In the absence of decoherence, the system evolves according to the Schrödinger equation, which leads to the superposition of states and the interference between them. However, when the system interacts with the environment, the entanglement between the system and the environment leads to the decay of the off-diagonal elements of the density matrix, which represent the quantum correlations between the states. As a result, the system loses its coherence and becomes classical, with well-defined values of the observables and no interference between the states. The decoherence process also leads to the emergence of a preferred basis of states, which are the pointer states that are stable under the environmental interaction and that correspond to the classical states of the system. Thus, the decoherence process explains the transition from the quantum world, which is characterized by superposition, entanglement, and interference, to the classical world, which is characterized by definite outcomes, separability, and objectivity.
\end{tcolorbox}

\section{Measurement-induced symmetry breaking}
\subsection{Spontaneous symmetry breaking and the emergence of order parameters}
The measurement process induces a spontaneous symmetry breaking of the USMS, which leads to the emergence of order parameters that characterize the broken symmetry. The order parameters are the expectation values of the observables that are being measured, which acquire non-zero values in the collapsed state. The spontaneous symmetry breaking is a non-perturbative effect that cannot be described by the perturbative expansion of the observables, and it requires a non-linear and self-consistent treatment of the measurement process.

\begin{tcolorbox}[colback=blue!5!white,colframe=blue!75!black,title=New terms]
\begin{description}
\item[Spontaneous symmetry breaking:] A phenomenon in which a system that is symmetric with respect to a group of transformations ends up in an asymmetric state, due to the instability of the symmetric state under small fluctuations. Spontaneous symmetry breaking is a key concept in particle physics and condensed matter physics, and it is responsible for the emergence of non-zero expectation values of the fields, such as the Higgs field and the order parameters of phase transitions.
\item[Order parameter:] A quantity that characterizes the degree of order or symmetry breaking in a system. The order parameter is zero in the symmetric phase and non-zero in the broken-symmetry phase, and it serves as a measure of the strength and direction of the symmetry breaking.
\end{description}
\end{tcolorbox}

\subsection{Reduction of symmetry and the emergence of classical structures}
The measurement-induced symmetry breaking leads to the reduction of the symmetry group $G$ of the USMS, and the emergence of classical structures and geometries that are invariant under the reduced symmetry group. The classical structures and geometries are the eigenstates of the observables that are being measured, and they correspond to the pointer states that are selected by the decoherence process. The emergence of classical structures and geometries is a hierarchical process that depends on the scale and resolution of the measurements, and it leads to the formation of a nested hierarchy of broken symmetries and emergent structures.


\begin{figure}[h]
\centering
\begin{tikzpicture}[scale=0.8, every node/.style={scale=0.8}]
\draw[thick, ->] (-5,0) -- (5,0) node[right] {Order parameter $\langle A \rangle$};
\draw[thick, ->] (0,-3) -- (0,3) node[above] {Potential $V(A)$};

\draw[dashed, color=blue] (-4,-2) to[out=90,in=180] (0,0) to[out=0,in=90] (4,-2);
\draw[fill] (0,0) circle (0.1) node[above right] {$\langle A \rangle = 0$};
\node[color=blue] at (-3,1) {Symmetric phase};

\draw[thick, ->, color=red] (0,0) -- (-2,-1) node[midway,above left] {Symmetry};
\draw[thick, ->, color=red] (0,0) -- (2,-1) node[midway,above right] {Breaking};
\draw[color=red] (-2,-1) to[out=-90,in=180] (0,-2) to[out=0,in=-90] (2,-1);
\node[color=red] at (3,-1) {Broken-symmetry phase};

\draw[pattern=north east lines, pattern color=green!50] (-3,-2.5) rectangle (-1,-1.5);
\node at (-2,-2) {Classical};
\node at (-2,-2.3) {Structures};

\draw[pattern=north east lines, pattern color=green!50] (1,-2.5) rectangle (3,-1.5);
\node at (2,-2) {Classical};
\node at (2,-2.3) {Structures};

\begin{scope}[xshift=7cm, yshift=1cm, node distance=1cm]
\node[draw, rectangle, fill=blue!20] (Symmetric) {Symmetric phase};
\node[draw, rectangle, fill=red!20, below=of Symmetric] (Broken) {Broken-symmetry phase};
\node[draw, rectangle, fill=green!20, below=of Broken] (Classical) {Classical structures};
\draw[thick, ->] (Symmetric) -- (Broken) node[midway, right] {Symmetry breaking};
\draw[thick, ->] (Broken) -- (Classical) node[midway, right] {Decoherence};
\end{scope}
\end{tikzpicture}
\caption{Schematic representation of the measurement-induced symmetry breaking and the emergence of classical structures. The potential $V(A)$ (blue dashed line) has a minimum at the origin, corresponding to the symmetric phase with $\langle A \rangle = 0$. The measurement of an observable $A$ on the USMS leads to the collapse of the state onto one of the eigenstates of $A$, which breaks the symmetry of the USMS (red arrows) and leads to the emergence of a non-zero order parameter $\langle A \rangle$ in the broken-symmetry phase (red solid line). The classical structures and geometries (green hatched boxes) emerge as the pointer states that are selected by the decoherence process, and they are invariant under the reduced symmetry group of the broken-symmetry phase. The legend on the right explains the relationship between the symmetry breaking, the decoherence, and the emergence of classical structures.}
\label{fig:symmetry_breaking}
\end{figure}
