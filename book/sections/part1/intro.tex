The nature of space and time is one of the most fundamental questions in physics and philosophy. The current understanding of space is based on the concepts of Euclidean geometry and Minkowski spacetime, which are postulated as \textit{a priori} structures. However, the origin and emergence of these structures from a more fundamental level of reality remain unexplained. 

In this thesis, we propose a novel framework for understanding the emergence of 3D space from an undifferentiated, symmetric morphological space using the tools of graph theory, quantum measurement theory, and morphic resonance.

\begin{tcolorbox}[colback=blue!5!white,colframe=blue!75!black,title=New terms]
\begin{description}
\item[Euclidean geometry:] The familiar geometry of flat space, characterized by the Pythagorean theorem. Assumed in classical physics.
\item[Minkowski spacetime:] The 4D spacetime of special relativity, with three spatial dimensions and one time dimension. Characterized by the invariant spacetime interval $ds^2 = -c^2 dt^2 + dx^2 + dy^2 + dz^2$.
\item[A priori:] Existing independently of experience or empirical evidence. Immanuel Kant argued that space and time are \textit{a priori} intuitions that structure our experience.
\end{description}
\end{tcolorbox}

\begin{tcolorbox}[colback=green!5!white,colframe=green!75!black,title=Question]
Why should we seek an explanation for the origin of space and time? Can't we just take them as given?
\tcblower
The postulation of absolute space and time has been challenged by the theories of relativity, which show that space and time are dynamical and intertwined. Moreover, the existence of spacetime singularities in general relativity and the conflict between general relativity and quantum mechanics suggest that our current understanding of spacetime is incomplete and needs to be revised at a fundamental level.
\end{tcolorbox}

\section{Significance and implications}
Our framework provides a unified description of quantum mechanics, general relativity, and the holographic principle, which are currently incompatible with each other. It also sheds light on the origin and evolution of the universe, including the Big Bang singularity, inflation, and the arrow of time. Moreover, our framework has important implications for the nature of consciousness, free will, and the mind-body problem, as it suggests a deep connection between observation, information, and the structure of reality.

\begin{figure}[h]
\centering
\begin{tikzpicture}[scale=0.8, every node/.style={scale=0.8}]
\draw[thick, ->] (0,0) -- (10,0) node[right] {Scale/Resolution};
\draw[thick, ->] (0,0) -- (0,6) node[above] {Complexity};

\draw[dashed, fill=blue!20] (1,5) to[out=-90,in=180] (5,1) to[out=0,in=-90] (9,5) -- (9,5.5) -- (1,5.5) -- cycle;
\node at (5,3) {Undifferentiated Symmetric};
\node at (5,2.5) {Morphological Space};

\draw[thick, ->, red] (5,1) -- (5,0.5) node[midway,right, text width=3cm, align=center] {Observation-induced Symmetry Breaking};

\draw[thick, ->, blue] (1,4.5) -- (1,5) node[midway,left] {Morphic Resonance};
\draw[thick, ->, blue] (9,4.5) -- (9,5) node[midway,right] {Structure Formation};

\draw[pattern=north east lines, pattern color=green!50] (2,0.2) rectangle (8,0.8);
\node at (5,0.5) {Emergent 3D Space};

\begin{scope}[xshift=12cm, yshift=2cm, node distance=1cm]
\node[draw, rectangle, fill=blue!20] (USMS) {USMS};
\node[draw, rectangle, fill=red!20, below=of USMS] (SB) {Symmetry Breaking};
\node[draw, rectangle, fill=blue!20, left=of SB] (MR) {Morphic Resonance};
\node[draw, rectangle, fill=blue!20, right=of SB] (SF) {Structure Formation};
\node[draw, rectangle, fill=green!20, below=of SB] (E3D) {Emergent 3D Space};
\end{scope}
\end{tikzpicture}
\caption{Schematic diagram of the emergence of 3D space from the undifferentiated morphological space (USMS). The interplay between observation-induced symmetry breaking (red) and morphic resonance (blue) leads to the formation of a low-dimensional pseudo-rectilinear subspace with Euclidean geometry (green). The legend on the right explains the meaning of the different elements in the figure.}
\label{fig:emergence}
\end{figure}